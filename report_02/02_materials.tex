\subsection{Materiales}

\begin{multicols}{2}
  \begin{itemize}
    \item Bioreactor Winpact One System Controller
    \item Autoclave HES de 75L
    \item 1L de medio YPD modificado (0.5\% peptona, 1\%extracto de levadura, 1\% glucosa) 
    \item Inóculo de un cultivo saturado de P. pastoris X-33.
    \item Micropipetas y puntas estériles. 
    \item Tubos de centrífuga estériles
    \item Espectrofotómetro
    \item Placas de 96 pocillos
    \item Centrifuga
    \item Guantes de Nitrilo
    \item Jeringa de 5mL
    \item alcohol 70\%

  \end{itemize}
\end{multicols}

\subsection{Métodos}

\subsubsection{Puesta en marcha del cultivo}

\begin{itemize}
  \item Iniciando se enciende el controlador y se da paso al agua para la circulación por el refrigerante. Se instala la manta llamada ''manta calefactora" para el control de temperatura del biorreactor. Se procede a revisar todos los sellos y mangueras conectadas para asegurar que no exista errores en el desarrollo del práctico. Se fijan los valores de temperatura en 30\textdegree y centrifugado en 150rpm del Biorreactor Winpact para proceder con el cultivo.
\end{itemize}

\subsubsection{Inoculación}
\begin{itemize}
  \item Se inicia limpiando el puerto de inoculación con alcohol 70\% para eliminar la contaminación. Se procede a tomar la jeringa de 5 mL, se inyecta 5 mL del inoculo del cultivo P. pastoris X-33. Los datos del monitor del Biorreactor al momento de inocular son de temperatura 30\textdegree , agitación 150 RPM, pH 6.05, DO 13.7\% y aireación a 0 LPM.
\end{itemize}

\subsubsection{Calculo de concentración de oxigeno disuelto (CL) desde valores de porcentaje de oxigeno disuelto (\%OD)}
\begin{itemize}
  \item Se procede a iniciar el biorreactor con el botón “Chart Stop'', con esto se empieza a registrar los datos que se muestran en el monitor, entregando un gráfico en donde se visualizan los datos de DO(Púrpura), pH(Blanca), Rpm(Verde) y \textdegree C(Roja).
\end{itemize}

\subsubsection{Determinación de velocidad de consumo de oxigeno disuelto}

\begin{itemize}
  \item Se procede a iniciar la aireación cambiando los datos de aireación a 1,5 LPM y agitación a 300 RPM hasta llegar a mostrar una semi curva en el gráfico. Al pasar una con cuatro minutos se observa que el DO llega a 93.2\%, con esto se vuelven a normalidad los datos anteriormente cambiados, aireación a 0 LPM y agitación a 150 RPM, provocando que la curva de la gráfica sé genere una pendiente en el DO hasta llegar al 11\%.
\end{itemize}

\subsubsection{Determinación de peso seco de la muestra}
\begin{itemize}
  \item Dado el puerto de muestra que tiene el biorreactor, se procede a sacar 20 mL de la muestra. Se pipeteo 8 mL de muestra en cuatro tubos de centrifugados, dejando en cada uno 2 mL de muestra los cuales son centrifugamos. Se procede a descartar el sobrenadante para quedarnos solo con el pelet e introducir buffer fosfato(pH7.5) para resuspender el pelet. Con un cestillo de aluminio(0,595 g) se agregan los 12 mL de la muestra, dejando secar a 75\textdegree por 24 horas para obtener el peso seco de la muestra en gramos.
\end{itemize}
