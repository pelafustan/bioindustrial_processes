\setlength{\parindent}{25pt}
\onehalfspacing
\spacing{1.5}

En los bioprocesos industriales, el objetivo principal es maximizar el crecimiento microbiano y/o 
producción de algún compuesto derivado por un microorganismo. Para lograr esto de manera eficiente, es 
necesario mantener un entorno adecuado en todo momento para dichos microorganismos \parencite{Johnssana2015}

El uso de  \emph{Pichia pastoris} como sistema de expresión para la producción de diversos tipos de proteínas recombinantes ha sido exitoso. Las investigaciones desarrolladas que involucran este sistema de expresión, han causado un gran impacto; no sólo en lo que se refiere a los altos niveles de expresión alcanzados, sino que también en la funcionalidad y actividad que presentan la mayoría de las proteínas expresadas \parencite{Macauley2015}.

La velocidad a la que las células consumen oxígeno en el biorreactor determina la velocidad a la que necesitan ser transferidas. Muchos factores afectan la demanda de oxígeno, el más importante de los cuales es el tipo de célula utilizada, en nuestro caso sera \emph{Pichia pastoris X-33} , el período de crecimiento del cultivo y la naturaleza de la fuente de carbono en el medio.

Un biorreactor es un aparato en el cual se transforman las sustancias sobre las que se trabaja, utilizando para ello células vivas o componentes celulares, como enzimas, capaces de producir una transformación bioquímica de una manera controlada. Un biorreactor tiene que ser diseñado de manera que los organismos vivos o las enzimas puedan emplearse en condiciones definidas y óptimas para que se produzcan los cambios bioquímicos buscados \parencite{Henzler2000}.
% 


