% \setlength{\parindent}{25pt}
% \onehalfspacing
% \spacing{1.5}

El crecimiento de un cultivo de bacterias están asociados a muchas variables que permiten la duplicación de estos organismos unicelulares como lo es la \emph{E. coli.} Dentro de aquellos factores se encuentran la temperatura, pH, y los nutrientes del medio. Por lo tanto dependiendo de las diferentes condiciones a las que están expuestas, presentarán diferenciación tanto fenotípica como de crecimiento del cultivo. 

Es posible obtener distintos crecimientos en distintos tipos de ambientes de incubación o crecimiento en varios tipos de medios de cultivo tanto sólido como líquido. Estos son los principales factores de crecimiento de las bacterias ya que en base a la abundancia de alimento que tengan, será su competitividad y desarrollo de la colonia. Sin embargo mayor cantidad de nutrientes no significa mayor crecimiento. Donde es necesario mantener una relación de volumen en la solución debido a la posibilidad de estresar las colonias reprimiendo su desarrollo por exceso de nutrientes. 

Para analizar y obtener el cálculo de microorganismos presentes en una muestra, existen una gran variedad de métodos, los cuales permiten obtener tanto su volumen, masa y cantidad de células. Dentro de ellas es posible considerar la cantidad de biomasa midiendo el peso su peso, o bien determinar su biovolumen a través de microscopía fotónica o epifluorescencia. Sin embargo un método menos costoso de realizar corresponde a espectrofotometría la cual mide los haces de luz que atraviesa el medio de cultivo y calcula su densidad
