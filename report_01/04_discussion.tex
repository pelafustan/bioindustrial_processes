Esta actividad de laboratorio tuvo por objetivo el crecimiento de un cultivo de bacterias E.coli en dos medios de cultivos diferentes. Medio LB y Medio SOC, los cuales poseen diferencias en cuanto a concentraciones de fuentes de carbono y otros minerales que influyen en el crecimiento del microorganismo. Así, en un principio lo que se debía lograr era la comparación de ambos medios de cultivo y determinar en cuál de ellos el microorganismo se desarrollada de mejor manera. El objetivo final del laboratorio es lograr la construcción de una curva de crecimiento de concentración de unidades formadoras de colonias del microorganismo (UFC/ml) en función del tiempo, las unidades formadoras de colonias constituyen una medida de la cantidad de microorganismo activo o vivo que se encuentra en una muestra. Una forma de medir la cantidad de material celular que crece a lo largo de un determinado tiempo es por medio de la densidad óptica o absorbancia, cuanto más material celular se encuentre en una muestra, menor es la cantidad de luz que atraviesa esta muestra misma debido a la dispersión de la luz por parte de las células en suspensión. La medición de la densidad óptica de las muestras se lleva a cabo haciendo uso de un instrumento denominado espectrofotómetro, de esta manera es posible medir la cantidad de material celular en suspensión en la muestra, sin embargo, esta medición no permite discriminar entre el material celular activo o células vivas y las células no vivas, por lo que, para efectos de poder obtener una curva de crecimiento en relación a la concentración de UFC en función del tiempo se debe construir una curva de calibración, esta grafica la densidad óptica en función de la concentración de UFC.
