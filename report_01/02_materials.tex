\subsection{Materiales}

\begin{multicols}{2}
  \begin{itemize}
    \item Medio estéril y medio LB estéril.
    \item Matraz Erlenmeyer de \SI{50}{\mL} con \SI{10}{\mL} de medio de cultivo.
    \item Glucosa, glicerol y sacarosa (sucrosa) estéril. 
    \item Micropipetas. 
    \item Placas multipocillos. 
    \item Espectrofotómetro de placas. 
    \item Agitador orbital con temperatura controlada.
  \end{itemize}
\end{multicols}

\subsection{Métodos}

\subsubsection{Primera parte}

Durante la primera parte se buscó establecer una aproximación para obtener valores de células viables a partir del registro de absorbancia a \SI{600}{\nm}.
\begin{itemize}
  \item A partir de un cultivo cuyo número de células viables es conocido (\si{\UFC\per\mL}), se generaron 10 diluciones seriadas en base a potencias de dos, las cuales fueron sometidas al espectrofotómetro a \SI{600}{\nm}. 
  \item El experimento se realizó en triplicado.
\end{itemize}

\subsubsection{Segunda parte}

Desde un cultivo
