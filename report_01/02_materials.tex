\subsection{Materiales}

\begin{multicols}{2}
  \begin{itemize}
    \item Medio estéril y medio LB estéril.
    \item Matraz Erlenmeyer de \SI{50}{\mL} con \SI{10}{\mL} de medio de cultivo.
    \item Glucosa, glicerol y sacarosa (sucrosa) estéril. 
    \item Micropipetas. 
    \item Placas multipocillos. 
    \item Espectrofotómetro de placas. 
    \item Agitador orbital con temperatura controlada.
  \end{itemize}
\end{multicols}

\subsection{Métodos}

Dentro de los elementos de riesgo lo más importante corresponde a la infección de los elementos estériles, para esto el exponer los materiales y herramientas al ambiente es de gran cuidado, de esta forma se trabaja en todo momento bajo el mechero con los tubos, placas y herramientas hacia abajo o cerrados en caso del matraz con el cultivo y los tubos de ensayo.


\subsubsection{Primera parte}

Durante la primera parte se buscó establecer una aproximación para obtener valores de células viables a partir del registro de absorbancia a \SI{600}{\nm}.
\begin{itemize}
  \item A partir de un cultivo cuyo número de células viables es conocido (\si{\ufc\per\mL}), se generaron 10 diluciones seriadas en base a potencias de dos, las cuales fueron sometidas al espectrofotómetro a \SI{600}{\nm}. En cada aplicación es necesario cambiar las puntas de la micropipeta con el fin de no contaminar las muestras.
  \item Graficar los datos de “DO” v/s “UFC/mL” (curva de calibración)
  \item El procedimiento de cada prueba es realizado tres veces con el fin de obtener el cálculo con el “coeficiente de determinación”. 
\end{itemize}

\subsubsection{Segunda parte}

\begin{itemize}
    \item A partir de un cultivo previo se debe inocular 0,2 mL de este en 20 mL de medio fresco. Se dispondrá de dos medios de cultivos distinto, medio LB (10g/L triptona, 5g/L extracto de levadura, 10g/L NaCl)l y medio SOC (20g/L Triptona, 5g/L extracto de levadura, 0.5g/L NaCl, 2.5mM MgC\textsuperscript{12}, 10mM MgSO\textsuperscript{4} 20mM glucosa). Cada media hora se debe medir la densidad óptica a 600nm y registrarla.
    \item Obtener los gráficos de “DO” vs “t”.
    \item Luego de obtener la curva de calibración es necesario generar una curva de crecimiento de “UFC/mL” v/s “t” 
\end{itemize}
